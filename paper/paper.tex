
\documentclass[runningheads,a4paper]{llncs}

\usepackage[ngerman]{babel}

\usepackage{graphicx}
\usepackage[table]{xcolor}

\usepackage{wrapfig}

%extended enumerate, such as \begin{compactenum}
\usepackage{paralist}

%put\ figures\ inside\ a\ text
%\usepackage{picins}
%use
%\piccaptioninside
%\piccaption{...}
%\parpic[r]{\includegraphics ...}
%Text...

%Sorts the citations in the brackets
%\usepackage{cite}

%for easy quotations: \enquote{text}
\usepackage{csquotes}

\usepackage[T1]{fontenc}
\usepackage[utf8]{inputenc}

%enable margin kerning
\usepackage{microtype}

%better font, similar to the default springer font
\usepackage[%
rm={oldstyle=false,proportional=true},%
sf={oldstyle=false,proportional=true},%
tt={oldstyle=false,proportional=true,variable=true},%
qt=false%
]{cfr-lm}
%
%if more space is needed, exchange cfr-lm by mathptmx
%\usepackage{mathptmx}


\usepackage[
	%pdfauthor={},
	%pdfsubject={},
	%pdftitle={},
	%pdfkeywords={},
	bookmarks=false,
	breaklinks=true,
	colorlinks=true,
	linkcolor=black,
	citecolor=black,
	urlcolor=black,
	%pdfstartpage=19,
	pdfpagelayout=SinglePage
]{hyperref}

%enables correct jumping to figures when referencing
\usepackage[all]{hypcap}

\usepackage[capitalise,nameinlink]{cleveref}
%Nice formats for \cref
\crefname{section}{Abschnitt}{Abschnitte}
\Crefname{section}{Abschnitt}{Abschnitte}
\crefname{figure}{Abb.}{Abb.}
\Crefname{figure}{Abbildung}{Abbildungen}

\usepackage{xspace}
%\newcommand{\eg}{e.\,g.\xspace}
%\newcommand{\ie}{i.\,e.\xspace}
\newcommand{\eg}{e.\,g.,\ }
\newcommand{\ie}{i.\,e.,\ }

%introduce \powerset - hint by http://matheplanet.com/matheplanet/nuke/html/viewtopic.php?topic=136492&post_id=997377
\DeclareFontFamily{U}{MnSymbolC}{}
\DeclareSymbolFont{MnSyC}{U}{MnSymbolC}{m}{n}
\DeclareFontShape{U}{MnSymbolC}{m}{n}{
    <-6>  MnSymbolC5
   <6-7>  MnSymbolC6
   <7-8>  MnSymbolC7
   <8-9>  MnSymbolC8
   <9-10> MnSymbolC9
  <10-12> MnSymbolC10
  <12->   MnSymbolC12%
}{}
\DeclareMathSymbol{\powerset}{\mathord}{MnSyC}{180}

%improve wrapping of URLs - hint by http://tex.stackexchange.com/a/10419/9075
\makeatletter
\g@addto@macro{\UrlBreaks}{\UrlOrds}
\makeatother

% correct bad hyphenation here
\hyphenation{op-tical net-works semi-conduc-tor}

\begin{document}

%Works on MiKTeX only
%hint by http://goemonx.blogspot.de/2012/01/pdflatex-ligaturen-und-copynpaste.html
%also http://tex.stackexchange.com/questions/4397/make-ligatures-in-linux-libertine-copyable-and-searchable
%This allows a copy'n'paste of the text from the paper
\input glyphtounicode.tex
\pdfgentounicode=1

\title{Web Technologien: Ransomware}
%If Title is too long, use \titlerunning
\titlerunning{Web Technologien: Ransomware}

%Single insitute
\author{Abwandner, Sebastian und Beham, Michael und Schmid, Pascal}
%If there are too many authors, use \authorrunning
%\authorrunning{First Author et al.}
\institute{
	Hochschule Augsburg \\
	An der Hochschule 1 \\
	86161 Augsburg
}


\maketitle

\begin{abstract}
Die letzten Jahre haben gezeigt, dass Ransomware für die IT-Sicherheit zum immer größer werdenden Problem wird. Obwohl schon vor Jahrzehnten die erste Ransomware entdeckt wurde, so ist es durch anonyme Zahlungsmethoden des Internets für Angreifer einfacher geworden Geld mit Malware zu verdienen. Diese Arbeit gibt einen Überblick über die Entwicklungen der letzten Jahre, die Ransomware im großen Stil möglich gemacht haben, über die technische Funktionsweise, psychologische Hintergründe und mögliche Gegenmaßnahmen. Zum Schluss soll ein Ausblick in die Zukunft nochmals deutlich werden lassen, welche Gefahr von Ransomware uns in den nächsten Jahren erwartet.
\end{abstract}

%%%%%%%%%%%%%%%%%%%%%%%%%%%%%%%%%%%%%%%%%%%%%%%%%%%%%%%%%%%%%%%%%%%%%%%%%%%%%%%
\section{Einleitung}\label{sec:intro}
In den letzten Jahren haben sich die bekannt gewordenen Fälle von Ransomware vervielfacht. Waren früher noch irreführende Apps der Hauptanteil der Malware, wurde 2012 mit den sogenannten Lockern erstmals Malware entdeckt, die großflächig die betroffenen Anwender zur Bezahlung einer Lösegeldsumme veranlasst hat. In den Jahren 2014 und 2015 wurden diese eher harmlosen Varianten, die nur die Benutzung des Geräts blockiert haben, durch die Crypto-Ransomware abgelöst. Diese verschlüsselt die Dateien des Benutzers und fordert zum Wiedererhalt eine Lösegeldsumme, die nicht-verfolgbar bezahlt werden soll. Schätzungen zufolge erbeuten die Hersteller von Crypto-Ransomware somit 100.000.000 Euro pro Jahr\cite{money}.

	\begin{figure}[h!]
		\centering
		\includegraphics[width=\linewidth]{img/ransom-evolution.png}
		\caption{Evolution Malware Quelle: \cite{evolution}}
		\label{fig:ransom-evo}
	\end{figure}
%%%%%%%%%%%%%%%%%%%%%%%%%%%%%%%%%%%%%%%%%%%%%%%%%%%%%%%%%%%%%%%%%%%%%%%%%%%%%%%

\section{Einteilung}
Aufgrund der sich teilweise überschneidenden Funktionen im Bereich der Malware und der teils sehr unscharfen oder auch falschen Verwendung der Worte in Medien aller Art, ist es schwierig eine allgemeingültige Definition dafür anzugeben. Im Rahmen dieser Arbeit wird sich jedoch an die Definitionen des Microsoft Protection Centers \cite{malware_pc} gehalten.

		\begin{itemize}
			\item \textbf{Bot:} Ein kleines, verstecktes Programm auf dem Rechner des Benutzers, oftmals vom Angreifer kontrolliert. Mehrere Bots, die über das Internet verbunden sind, nennt man ein Botnetz.
			\item \textbf{Botnetz:} Mehrere Kopien des­sel­ben Bots sind auf mehreren Rechnern installiert und werden von einem Angreifer kontrolliert. Mehrere Bots auf einer großen Anzahl infizierter Rechner nennt man Botnetz.
			\item \textbf{Malware:}	Kurzform für \glqq malicious software\grqq. Überbegriff für Software, die ungewollte Aktivitäten auf einem Rechner durchführt, wie etwa Daten stehlen, Dateien verschlüsseln oder Spam verschicken.
			\item \textbf{Ransomware:} Eine Art Malware, die den Benutzer daran hindert, das Gerät zu benutzen oder Dateien verschlüsselt. Der Benutzer wird darüber informiert, dass er Geld bezahlen, eine Befragung oder eine sonstige Aktion durchführen muss, bevor er die Kontrolle wiedererlangt.
			\item \textbf{Trojaner:} Nach dem Trojanischen Pferd aus der Sage benannt, verhält sich diese Art von Malware unauffällig, um nicht entdeckt zu werden. Im Gegensatz zu Würmern oder Viren verbreitet sich ein Trojaner nicht von selbst weiter, sondern täuscht den Benutzer, um von diesem heruntergeladen oder installiert zu werden.
			\item \textbf{Virus:} Malware, die sich selbst weiterverbreitet, indem es reguläre Programme befällt und sich auf andere Systeme und Netzwerke weiterverbreitet.
			\item \textbf{Wurm:} Ein Wurm unterscheidet sich vom Virus dadurch, dass er mithilfe eines Trägers verbreitet wird. Dies kann etwa durch E-Mails, Instant Messaging, File-Sharing, Netzwerklaufwerke, Wechsellaufwerke und Ähnliches geschehen.
		\end{itemize}
Diese Einteilung gibt einen kurzen Überblick über die gebräuchlichsten Begrifflichkeiten, ist aber nicht abschließend. \\
Einer klaren Kategorisierung steht auch entgegen, dass Schädlinge meist in mehrere Kategorien fallen und sich diese teilweise überschneiden.

\section{Aktuelle Ransomware}
Alle hier beschriebenen Ransomware sind innerhalb des letzten Jahres erschienen und beschreiben somit den zum Schreiben des vorliegenden Papers aktuellen Stand der Technik.

\subsection{TeslaCrypt}
TeslaCrypt war der erste der modernen Ransomware, der von den Medien aufgegriffen wurde und somit der Öffentlichkeit bekannt gemacht wurde. Ende 2015\cite{tesla:entdeckt} wurde dieser entdeckt, jedoch wurde bereits kurze Zeit später ein Tool veröffentlicht\cite{tesla:geknackt}, mit dem betroffene Anwender ihre Daten retten konnten.
Version 3 der Ransomware wurde von den Angreifern grundlegend verändert. So wurde der Algorithmus zum Schlüsselaustausch erneuert, was es vorerst unmöglich gemacht hat, den privaten Schlüssel der Ransomware aus den verschlüsselten Dateien herzustellen\cite{tesla:version3}\cite{tesla:version3_2}.
Die letzte Version 4 kann zudem Dateien größer als 4 GB verschlüsseln\cite{tesla:version4}.....
		
\subsection{Petya}
\subsection{Locky}
\subsection{KeRanger}

\section{Faktoren von Ransomware}
Um den betroffenen Nutzer zur Zahlung der Lösegeldsumme zu überreden, spielen den Angreifern mehrere Faktoren in die Hände. Diese sind vermutlich eher unbewusst eingebaut, da sich die Angreifer wahrscheinlich nicht noch zusätzlich Psychologen in ihr Team holen, um durch vermehrte psychologische Faktoren möglichst viel Gewinn zu erzielen. Im Bereich Internationalisierung \cite{faktoren:l18n}, Usability und grafische Aufbereitung \cite{faktoren:grafik} \cite{evolution} kann jedoch davon ausgegangen werden, dass hier auf professionelle Hilfe zurückgegriffen wurde.

\subsection{Psychologische Aspekte}

Interessant ist die Betrachtung der verschiedenen psychologischen Faktoren, mit deren Hilfe sich Locker- und Crypto-Ransomware, bewusst oder unbewusst, der menschlichen Psyche bedienen:
Die Locker-Ransomware \glqq Lockdroid.G\grqq{} (siehe Abb.~\ref{fig:lockdroid}), die Android-Smartphones angreift, täuscht Anwender. Kognitive Mechanismen nutzen gerne Abkürzungen, um so die gedankliche Effizienz zu steigern. Dies ist der Grund dafür, dass Täuschungen wie etwa das Abbilden von Logos von offiziellen Strafverfolgungsbehörden dazu führt, dass Menschen die Legitimität der Meldung nicht anzweifeln. 

\begin{wrapfigure}{r}{0.3\textwidth}
  \begin{center}
    \includegraphics[width=0.48\textwidth]{img/android_locker.png}
  \end{center}
  \caption{Locker am Beispiel \glqq Lockdroid.G\grqq{} \cite{evolution}}
  \label{fig:lockdroid}
\end{wrapfigure}

Das Elaboration Likelihood Model geht davon aus, dass es zwei Wege zur Überzeugung gibt. Überzeugung durch die zentrale Verarbeitung der Mitteilung beschreibt die Abwägung und Qualität der Argumente. Hierbei wird also mit wohl überlegten Argumenten gearbeitet, um das Opfer zur Zahlung des Lösegelds zu bewegen. Im Falle von \glqq Lockdroid.G\grqq{} wäre das etwa die Tatsache, dass die Polizei das Smartphone \glqq aus Gründen der Sicherheit\grqq{} sperrt.\\
Die periphere Verarbeitung der Mitteilung beschreibt dagegen eher konditionierte Verhaltensweisen. Es wird davon ausgegangen, dass negativ oder positiv konnotierte Stichwörter zum Ergebnis führen. Als Beispiele hierfür kann z.B. das Logo der Polizei dienen oder auch die Anzeige des Landes, IP-Adresse und Stadt des Benutzers, die beim Opfer zur Angst führen, dass er etwas Illegales getan habe. \\

Wird noch angezeigt, dass der Nutzer illegal Dateien wie Musik oder Filme herunter geladen habe, führt dies dazu, dass er aus Angst vor sozialer Stigmatisierung nicht bei Bekannten oder der Polizei um Hilfe bittet.\\

Während Locker-Ransomware eher mit den psychologischen Faktoren innerhalb der Nachricht arbeitet, benutzt Crypto-Ransomware die Empfindungen den verschlüsselten Daten gegenüber und welchen Schaden das Opfer hätte, diese wertvollen Daten zu verlieren.\\

Zum einen enhalten viele Crypto-Ransomware einen Countdown, nach dessen Ablauf die Daten endgültig verloren seien sollen. Der Faktor Zeit wurde schon in Tests auf irrationale Entscheidungen hin untersucht. Zudem kann zum Countdown die Lösegeldforderung höher werden, je weiter die Zeit voran schreitet. Das setzt das Opfer so unter Druck, dass dieses bereitwillig bezahlt.\\
Das Ellsberg-Paradoxon der Entscheidungstheorie beschreibt das Verhalten von Menschen, die sich in einer Situation mit umgewissen Ausgang befinden. Wenn sie mit 2 Möglichkeiten konfrontiert sind, deren eine Möglichkeit einen Gewinn darstellt, dessen Wahrscheinlichkeit aber nicht abzusehen ist und der anderen Möglichkeit, dass sie etwas verlieren, aber der Verlust von der Wahrscheinlichkeit abzuschätzen ist, so nehmen Menschen eher den Verlust in Kauf.\\
Im Falle von Crypto-Ransomware haben die Menschen es mit zwei negativen Möglichkeiten zu tun: Einmal können sie nicht sicher sein, ob sie durch Bezahlung ihre Daten wieder erhalten. Auf der anderen Seite können sie noch weniger sicher sein, wieweit der Verlust ihrer Daten sie beeinflusst. Auch hier ist die Tendenz eher Richtung bezahlen und auf Wiedererhalt der Daten hoffen.

\begin{figure}[h!]
	\centering
	\includegraphics[width=\textwidth]{img/cryptolocker.png}
	\caption{Crypto-Ransomware am Beispiel \glqq Cryptolocker\grqq\cite{evolution}}
	\label{fig:cryptolocker}
\end{figure}


\subsection{Preis und Bezahlsysteme}

Bereits die AIDS Ransomware hatte als Lösegeldsumme circa 300 \$ (Inflation eingerechnet), was dem Preis aktueller Ransomware entspricht\cite{evolution}.\\ 
Die durchschnittlichen Einkommen der unterschiedlichen Länder finden sogar Einzug in Ransomware: Um das Opfer zur Zahlung zu bewegen, darf sich die Lösegeldforderung nicht auf einem Level befinden, welches die Einkommensgrenze überschreitet. So wird in Ländern wie beispielsweise Indien oder Rumänien etwa ein anderer Betrag gefordert, als in den USA und Deutschland. Technisch wird dies dadurch gelöst, dass sich ein infizierter Rechner bei dem Command-and-Control-Server der Angreifer meldet und dieser dadurch anhand der IP-Adresse eine grobe Zuordnung nach Ländern treffen kann.\\

Eine weitere Anpassung beim Preis wird bei der Unterscheidung zwischen Privatperson und Firma getätigt: Wird festgestellt, dass eine Firma betroffen ist, so werden einige Tausend Dollar verlangt, um die Daten zurückzuerhalten. Sicherheitsexperten haben heraus gefunden, dass der ideale Punkt, den Firmen noch bereit sind zu zahlen, und bei dem die Polizei noch keine Ermittlungen beginnt, bei 10.000 \$ liegt\cite{sweetspot}.\\

Um den Strafverfolgungsbehörden zu entgehen, ist es für die Ransomware-Community notwendig, anonym bezahlt zu werden, da sich so keine Spuren nachverfolgen lassen. War dies zur Zeiten von AIDS noch ein anonymes Postfach in Panama, so hat sich das im Digitalzeitalter geändert. Anfangs wurde noch der Anruf einer kostenpflichtigen Nummer oder die Bezahlung mit Bezahlgutscheinen wie Pasafecard oder CashU präferiert, inzwischen ist der Fokus bei Kryptowährungen wie Bitcoin. Diese ermöglichen es anonym zu bleiben und den Geldfluss nicht verfolgen zu können.\\

Interessant ist auch, dass je nach Ransomware-Art andere Bezahlsysteme zum Tragen kommen. Während durch Lockern der Rechner gesperrt ist, können etwa keine Bitcoins online gekauft werden, um mit diesen der Lösegeldforderung nachzukommen. In diesem Fall wird dann auf Bezahlgutscheine zurück gegriffen, die es einfach und schnell an vielen Tankstellen und Supermärkten zu kaufen gibt. Dagegen ist bei Crypto-Ransomware der Rechner noch bedienbar, also können noch Bitcoins erworben werden. Je nach Ransomware wird sogar ein Video als Bedienungsanleitung dazu gezeigt.

\subsection{"`Probepackung"'}

Um dem Opfer zu demonstrieren, dass es die verschlüsselten Dateien wieder entschlüsselt bekommt, sollte es den Betrag bezahlen, bietet manche Ransomware an, dass eine handvoll zufällig ausgewählte Dateien des Rechners entschlüsselt werden. Dies steigert das Vertrauen der Opfer und erhöht die Chancen einer Bezahlung.

\section{Kryptographie}
\label{sec:sym_verschl}

Die symmetrische Verschlüsselung folgt dem Konzept, das sowohl für die Ver- als auch die Entschlüsselung jeweils die gleichen Schlüssel verwendet werden. \\
Abbildung~\ref{fig:sym_verschl} zeigt eine schematische Darstellung.

\begin{figure}[h!]
	\centering
	\includegraphics[scale=0.22]{img/SymKrypto.png}
	\caption{Symmetrische Verschlüsselung mit identischen Schlüsseln \\
	Bananenfalter (Public Domain) \\ \url{https://commons.wikimedia.org/wiki/File:Orange_blue_symmetric_cryptography_de.svg}
	}
	\label{fig:sym_verschl}
\end{figure}

Es gibt eine Reihe an unterschiedlichsten Algorithmen und Verfahren für symmetrische Verschlüsselung, allerdings hat sich mit \texttt{AES} ein Verfahren als de-facto Standard für verschiedenste Anwendungsbereiche etabliert.

\texttt{AES} ist hierbei eine Abkürzung für \texttt{Advanced Encryption Standard} und wird meist synonym für den \texttt{Rijndael}-Algorithmus verwendet. Dieses Verfahren ist heute so verbreitet, dass es beispielsweise in modernen Prozessoren direkt als Maschinenbefehl zur Verfügung steht \cite{crypto:aes_intel}. 
\\

Bei Ransomware ist es während der Phase, in der die eigentliche Dateiverschlüsselung durchgeführt wird, wichtig, nicht vom Opfer erkannt zu werden. Allerdings würde es wahrscheinlich auf unkundigen Nutzern auffallen, wenn der Computer auf Eingaben nur noch verzögert reagiert, wie es typischerweise geschieht, wenn die \texttt{CPU} im Hintergrund mit rechenintensiven Aufgaben, zu denen auch Verschlüsselungsoperationen gehören, ausgelastet ist.

Mittels der Hardwarebeschleunigung wird zwar die \texttt{CPU}-Belastung reduziert, dennoch ist oftmals eine Verlangsamung des Computers festzustellen, vor allem bedingt durch die großen Datenmengen die vom den angeschlossenen Speichergeräten gelesen und verschlüsselt geschrieben werden müssen.

Sowohl für den Entwickler als auch den Betreiber der Ransomware ist es hier vorteilhaft eine gewisse Lastbegrenzung, sowohl für Speicher als auch \texttt{CPU}, einzubauen bzw. zu verwenden.
\\

Der größte Nachteil beim Einsatz von symmetrischer Verschlüsselung ist allerdings die Eigenschaft, dass sowohl für die Ver- als auch für die Entschlüsselung der gleiche Schlüssel verwendet wird. Prinzipbedingt muss der Schlüssel während der Verschlüsselung einer Datei im \texttt{RAM} gehalten werden und kann somit von anderen Programmen ausgelesen werden.

Wird ein Schlüssel für mehrere oder auch alle Dateien verwendet, so stellt das Auslesen des verwendeten Schlüssels aus dem \texttt{RAM} ein gute Möglichkeit dar, ohne Zahlung die verschlüsselten Dateien wird zu decodieren. 

Diese prinzipielle Schwäche einer symmetrischen Verschlüsselung lässt sich durch die Verwendung asymmetrischer Verschlüsselung umgehen.

\subsection{Asymmetrische Verschlüsselung}
\label{sec:asym_verschl}

\begin{figure}[h!]
	\centering
	\includegraphics[scale=0.25]{img/AsymKrypto.png}
	\caption{Asymmetrische Verschlüsselung \\
	Bananenfalter (Public Domain) \\ \url{https://commons.wikimedia.org/wiki/File:Orange_blue_symmetric_cryptography_de.svg}}
	\label{fig:asym_verschl}
\end{figure}

Asymmetrische Verschlüsselung basiert auf zwei Schlüsseln, einem öffentlichen und einem privaten. Diese Schlüssel gehören mathematisch zusammen und werden für unterschiedliche Zwecke verwendet.

Soll ein Klartext verschlüsselt werden, so wird hierfür der öffentliche Schlüssel des Empfängers verwendet. Die Entschlüsselung muss dann mit dem zugehörigen privaten Schlüssel erfolgen, eine Entschlüsselung mit dem öffentlichen Schlüssel ist nicht mehr möglich.

Es existieren mehrere genaue Verfahren für asymmetrische Verschlüsselung. Die Bekanntesten sind hier das \texttt{RSA}-Verfahren und das \texttt{Elgamal}-Verfahren.\\
Beide Verfahren unterscheiden sich zwar mathematisch stark voneinander, da \texttt{RSA} auf dem Problem der Primfaktorzerlegung basiert, während sich \texttt{Elgamal} auf die Schwierigkeit des diskreten Logarithmus stützt. \\
Hinsichtlich ihrer Verwendung und Sicherheit im Umfeld von Ransomware sind sich beide Verfahren ebenbürtig, weshalb im weiteren nur das \texttt{RSA}-Verfahren behandelt wird.
\\

Auf Grund der Komplexität der Berechnungen ist \texttt{RSA} im Vergleich zu \texttt{AES} langsam \cite{crypto:aes_rsa_benchmark}. Deshalb wird es nur für verhältnismäßig kurze Klartexte eingesetzt (< 1 KB), auch da die maximale Länge des Klartexts von der Länge des verwendeten Schlüssels abhängt. %TODO: Nachweis
\\

Der große Vorteil von asymmetrischer Verschlüsselung liegt im sicheren Schlüsselaustausch, der nicht auf einen bereits gesicherten Kanal angewiesen ist, sondern auch über ein unsicheres Medium, beispielsweise über das Internet erfolgen kann, ohne dass es für einen Angreifer möglich ist die beteiligten Schlüssel zu belauschen oder zu berechnen. %TODO: MitM




\subsection{Hybride Verschlüsselung}
\label{sec:hybride_verschl}

Hybride Verschlüsselung kombiniert alle Vorteile von symmetrischer Verschlüsselung (Geschwindigkeit, Einfachheit) mit allen Vorteilen von asymmetrischer Verschlüsselung (Schlüsseltausch, Sicherheit).
Hierfür werden zufällige Schlüssel erzeugt, meist lange Zufallszahlen, die für die symmetrische Verschlüsselung verwendet werden. Diese Schlüssel werden Sitzungsschlüssel genannt und werden in der Regel nur einmalig verwendet (ein sog. One-Time-Pad). Nach der Verwendung werden die Sitzungsschlüssel asymmetrisch verschlüsselt und mit den codierten Daten zusammen übertragen.

Auf diese Weise lassen sich auch große Mengen an Daten effizient mittels asymmetrischer Kryptographie übertragen.


\section{Zahlung}
Eine weitere Komponente warum heutige Ransomware so erfolgreich ist, stellt die Möglichkeit einer anonymen Bezahlung dar, vor allem in Hinsicht auf die Anonymität des Zahlungsempfängers. \\
Waren beim BKA-Trojaner noch kartengebundene Zahlungssysteme wie Paysafecard und Ukash im Einsatz, verlagerte sich die Zahlung bei heutiger Crypto-Ransomware vor allem auf Bitcoin. Dies dürfte auch mit der heutigen Popularität von Bitcoins und dem Erreichen breiter Schichten der Bevölkerung. Trotzdem bietet beispielsweise Petya eine genau Erklärung zum Erwerb und zur Weitergabe der Bitcoins, um auch unkundige Nutzer die Zahlungen zu ermöglichen. \cite{petya:infect}

\subsection{Paysafecard}
Paysafecard ist ein anonymes Zahlungsmittel, bei dem der Kunde an einer Verkaufsstelle (z.B. Automat, Tankstelle, Einzelhandel) eine Guthabenkarte mit einem Wert zwischen 10 € und 100 € erwirbt. Das Guthaben kann mittels der Kenntnis einer aufgedruckten PIN weitergeben und ausbezahlt werden. \\
Durch die fehlende Kontrolle beim Erwerb der Karten und der Möglichkeit die Guthaben automatisiert europaweit auszubezahlen, war für die Angreifer eine hinreichende Anonymität anscheinend sichergestellt. 

Seitens der Politik wurde immer wieder diskutiert, ob derartige Zahlungssysteme nicht reguliert oder verboten werden sollten, allerdings konnte hierüber bis heute keine abschließende Einigung erzielt werden. \cite{paysafecard}

Bei moderner Ransomware spielen guthaben-basierte Zahlungsmittel keine Rolle mehr.

\subsection{Bitcoin}
Bitcoin wird von moderner Crypto-Ransomware als bevorzugtes Zahlungsmittel verwendet. Bitcoin wurde von Anfang an als dezentrales, anonymes Zahlungsmittel konstruiert. Hinsichtlicher der Nutzung durch Ransomware bietet sich hier der Vorteil, dass sowohl die Adressen (vergleichbar mit einem Bankkonto) ohne zentrale Stelle automatisiert erzeugt werden können, als auch die Möglichkeit mit Hilfe von Schnittstellen diese Adressen ebenfalls automatisiert auf Zahlungseingänge überwachen zu können. 

Typischerweise wird für jeden "`Nutzer"' der Ransomware eine eigene Bitcoin-Adresse erzeugt, um die Zahlungen eindeutig zuordnen zu können und um eine Nachverfolgung der Zahlungen, die bei Bitcoin prinzipbedingt öffentlich sichtbar sind, zu erschweren. 

Für weiterführende Informationen bezüglich Bitcoin empfiehlt sich das Paper von Satoshi Nakamoto. \cite{bitcoin}


\section{Tor}
Zur Absicherung der Kommunikation mit den Kontroll-Servern (C\&C-Server), die für die Auslieferung für die spezifischen Angriffsbefehle und auch die Hinterlegung der Schlüssel notwendig sind, wird typischerweise Tor als Anonymisierungsschicht verwendet, gerne auch in Kombination mit öffentlichen Gateways zum Internet. \\
Tor ermöglicht über die mehrfache Weiterleitung und den sog. versteckten Diensten, dass weder die Adressen noch der Standort der Server bekannt wird. \cite{tor:hidden}

Für die Angreifer ist hierbei besonders die Möglichkeit der anonymen Bereitstellung der Dienste essenziell, da so verhindert wird, dass die Server vom jeweiligen Anbieter offline genommen werden.


\section{Prävention und Gegenmaßnahmen}
\subsection{Aktuelle Patchlevel}

	 Grundsätzlich sollten, um die allgemeine Sicherheit eines Systems zu verbessern, jegliche Software, sowohl das Betriebssystem wie auch die Anwendungssoftware, auf dem aktuellen Stand gehalten werden. Ein besonders großes Risiko besteht bei Software, die zum Öffnen von Inhalten aus dem Internet verwendet werden, wie z.B. Web-Browser, Browser-Plugins, E-Mail-Programme, PDF-Betrachter und Büroanwendungen. \cite{bsi:ransome}

Dies gilt in besonderem Maße für Systeme auf denen Windows betrieben wird, da diese für die Erpresser die größte Verbreitung ihrer Software ermöglichen. Dies soll keinesfalls bedeuten, dass alternative Systeme weniger gefährdet sind, allerdings das Risiko bei Windows basierten Systemen erhöht sein könnte.

Da allerdings sowohl für MacOS als auch für Linux ebenfalls Ransomware existiert, die beispielsweise durch die allgegenwärtigen Lücken in WordPress den Server infizieren, kann als Maßzahl hauptsächlich die Verbreitung einer Kombination von Systemen angesehen werden.
	 
\subsection{Angriffsfläche minimieren}
Zudem kann die Angriffsfläche reduziert werden, indem nicht benötigte Software deinstalliert wird, wodurch auch die Anzahl der Software, die auf dem neuesten Stand gehalten werden muss, verringert wird.
	
Auch sollte nach Möglichkeit systemweite die Skript-Ausführung deaktiviert werden, da Ransomware teilweise über E-Mail-Anhänge in Form von Javascript und Windows Script Host-Skripten verteilt wird. Somit kann der Anhang nicht auf dem System ausgeführt werden und eine Infektion wird effektiv verhindert. \cite{bsi:ransome}
	
Zudem ist es sinnvoll, nicht benötigte Browser-Plugins zu deinstallieren (z.B. Flash, Java, Silverlight) \cite{bsi:ransome} , da diese bekanntermaßen häufig über Sicherheitslücken verfügen, was alleine schon an den häufigen Sicherheits-Patches der Hersteller festgemacht werden kann.  
	
\subsection{Makros deaktivieren}

Makros in z.B. Microsoft Office stellen grundsätzlich ein Sicherheitsrisiko dar. Ransomware kann sich z.B. in unscheinbar wirkenden Excel- oder Word-Dateien als Makro verstecken. \\
Bei aktuellen Office-Produkten ist das automatische Ausführen von Makros zwar deaktiviert, allerdings verwenden viele Firmen als Teil ihrer Arbeitsumgebung hauseigene Makros, weshalb hier oftmals die Ausführung standardmäßig erlaubt ist. 
	
\subsection{Text-Mail erzwingen}
Dem BSI zufolge ist es ratsam, die HTML-Darstellung von E-Mails zu deaktivieren. Dies kann hilfreich sein, da für die Darstellung solcher E-Mails dieselben Mechanismen benötigt werden, wie sie in Web-Browsern nötig sind. Dies bringt zum Einen weitere Komplexität in die E-Mail-Software und somit potentiell weitere Sicherheitslücken, zum Anderen ermöglicht HTML eine Reihe anderer Angriffsvektoren. \\
Diese können natürlich nicht ausgenutzt werden, wenn man entsprechende Komponenten nicht nutzt. \cite{bsi:ransome}
	
	
\subsection{Greylisting / Tarpitting}
	
Um einer Infektion durch E-Mail zuvorzukommen, hilft es den Angriffsvektor durch Empfang von Spam-Mails generell zu unterbinden. Die Erfahrungen der letzten Jahre haben gezeigt, dass Blacklisting generell eine eher schlechte Idee ist. Manche Blacklists sperren gleich komplette Netzbereiche eines ISPs, wenn aus dessen IP-Adress-Bereich Spam verschickt wird.

Greylisting ist eine 2003 erfundene Methode, die einen alternativen Ansatz bietet: Der empfangende Mail-Server schickt dem potenziellen Spammer eine Nachricht, dass der Service zurzeit überlastet ist und der Sender sich zu einem späteren Zeitpunkt nochmals melden soll. Dieses standardkonforme Verhalten wird aber von Spammern meist nicht implementiert und so melden sich nur reguläre Mail-Server nach einer Zeit wieder. \cite{greylisting}

Einen Schritt weiter geht \glqq Tarpitting\grqq{}, bei dem ähnlich wie bei der namensgebenden Teergrube die Geschwindigkeit verlangsamt wird. Der Spammer möchte soviele Mails wie möglich in kurzer Zeit versenden. Greylisting verjagt die Spammer zwar, aber lässt diese noch ungehindert ihr Werk tun. Die Tarpit reguliert den Mailempfang auf eine möglichst geringe Geschwindigkeit herunter, um so den Spammer lange im Sendezustand zu lassen. Solang eine Verbindung offen ist, wird er nur eine begrenzte Anzahl weiterer öffnen. So wird beim Spammer eine Last erzeugt und dieser am massenhaften Versenden weiterer Mails gehindert.
	
\subsection{Serverseitige Mailfilter}

Es gibt allerdings noch weitere Möglichkeiten die Angriffsvektoren zu verringern. Eine ist es, alle ausführbaren Dateien (exe, src, chm, bat, com, msi, jar, cmd, hta, pif, scf) grundsätzlich zu löschen, oder zumindest in Quarantäne zu verschieben. \cite{bsi:ransome} \\
Das selbe gilt auch für verschlüsselte Archive und Office-Dateien mit Makros. Es ist durchaus möglich, dass dies nicht praktikabel ist. Dennoch sollten potenziell gefährliche Anhänge als solche prominent markiert werden. \cite{bsi:ransome}

Neben Greylisting kann man noch das SPF\footnote{Sender-Policy-Framework} nutzen, welches bereits die Annahme von nicht legitimen E-Mails reduziert. Auf Grund der mangelnden, flächendeckenden Verbreitung von SPF führt es allerdings häufig zu False-Negativ-Resultaten.\\
Zudem sollte der Mail-Server so konfiguriert sein, dass dieser keine Mails von internen Mail-Adressen vom externen Netzwerk entgegen nimmt. Dies verhindert, dass vermeintliche Mails von Arbeitskollegen gefälscht werden können, um so eine Legitmität vorzutäuschen.
	
\subsection{Netzlaufwerke sichern}

Eine zentrale Datenhaltung ist heute unerlässlich, schon aus dem Grund, dass zentrale Backups erstellt werden können. Dies kann allerdings zu einer Gefahr werden, da bereits Ransomware existiert, die in der Lage ist Netzlaufwerke, auf welche der Benutzer Schreibrechte hat, zu verschlüsseln. \\
Dies kann besonders großen Schaden verursachen, da nicht nur die Daten des Benutzers, sondern auch Daten von anderen Benutzern, deren Rechner nicht infiziert sind, auch verschlüsselt werden. 

Deshalb lautet die Empfehlung des BSI die Daten von z.B. bereits beendeten Projekten serverseitig auf schreibgeschützt zu setzen und somit ändernde Zugriffe durch die Ransomware unterbunden werden. \cite{bsi:ransome} \\
Hierbei sollte allerdings beachtet werden, dass dies nur für Dateien, die nicht mehr bearbeitet werden müssen anwendbar ist. Somit wären z.B. Dateien eines aktiv laufenden Projektes auf diese Weise nicht schützbar, da dies im Alltag einfach nicht praktikabel wäre.
	
\subsection{Netzwerke segmentieren}
	Eine weitere Möglichkeit ist es das Netzwerk zu unterteilen, um z.B. den Zugriff auf Projektdaten nur den erforderlichen Teams zu erteilen. Somit kann sichergestellt werden, dass nur die Daten des Teams in Mitleidenschaft gezogen werden, in welchem ein Rechner infiziert wurde. Zudem sollte darauf geachtet werden, dass das Administrator-Konto entsprechend gesichert ist, denn hiermit steht und fällt das komplette Konzept. \cite{bsi:ransome}
	
\subsection{Zugänge sichern}
Wie bereits erwähnt, ist es wichtig, dass Administrator-Zugänge gut gesichert sind. Dies gilt nicht nur für spezielle, sondern für alle Zugänge. \\
Manche Ransomware ist unter anderem dazu in der Lage, sich über gekaperte Remote-Zugänge zu verbreiten. Solche Zugänge sollten wenn möglich immer über VPNs abgesichert werden und zudem mit einer Zwei-Faktor-Authentifizierung versehen werden. Zudem ist es möglich ein höheres Sicherheitslevel durch Quell-IP-Filter zu erreichen. \cite{bsi:ransome}
	
\subsection{Aktueller Virenschutz}
Die Antiviren-Hersteller benötigen circa 12 Stunden vom Zeitpunkt der Erkennung einer neuen Malware bis zum fertigen Update für ihre Softwarelösung. Um gegen eine Infizierung durch Ransomware gewappnet zu sein, ist es also anzuraten, möglichst täglich auf notwendige Updates hin zu überprüfen. Dies ist in nahezu allen Antiviren-Lösungen auch automatisch einstellbar.

\subsection{Nutzerschulung}
Technische Lösungen sind immer nur ein Mittel zur Schadensbegrenzung, aber nie eine völlige Sicherheit vor Malware. \\
Wie bereits dargelegt, versucht Malware sich durch gefälschte E-Mails zu verbreiten, die dem Benutzer suggerieren, dass in einer angehängten Datei ein für ihn wichtiger Inhalt wäre. Zum Schutz vor Malware sollte hier den Benutzern eingeschärft werden, dass sie nur Anhänge von ihnen bekannten Personen öffnen. \\
Nutzerschulungen sind leider kostspielig und müssen in regelmäßigen Abständen wiederholt werden, was oftmals in der Praxis der Grund ist, auf derartige Maßnahmen zu verzichten.


\subsection{Backups / Datensicherungen}

Falls trotz aller getätigter Anstrengungen doch eine Infizierung erfolgt, ist es unabdingbar aktuelle Backups zu besitzen. Diese sollten in regelmäßigen Abständen und automatisch erfolgen, da sie sonst erfahrungsgemäß irgendwann nicht mehr durchgeführt werden. \\
Da Ransomware meist auch versucht auf Netzlaufwerke zuzugreifen, ist es ebenso wichtig, dass die Backups auf einem WORM-Medium (Write Once Read Many, etwa Bandspeicher)  gespeichert werden. So kann verhindert werden, dass die Ransomware auch die Backups verschlüsselt. \\
Ansonsten gilt wie üblich, dass Backups getestet werden müssen, um ihre Konsistenz und Wiederherstellbarkeit zu überprüfen und zu gewährleisten.

Eine weitere Möglichkeit ist es Versionierungssysteme wie beispielsweise \textit{Git} als Backupsystem für Dokumente zu verwenden. \\
Hierbei ist es wichtig, dass Git nicht nur lokal verwendet wird, sondern auf jeden Fall zusammen mit einem Remote-Server, auf den die Änderungen \textit{gepusht} werden. Würde das Git nur lokal verwendet werden, würde es absolut keinen Schutz bieten, da die Ramsomware einfach das ganze Repository verschlüsselt und somit unbrauchbar machen würde.\\
Ein Alternative hierfür wäre z.B. \textit{bup}. Diese Software baut auf das Paketformat von Git, ermöglicht es allerdings auch sehr große Dateien effizient und schnell zu sichern. \cite{bup}
	
	
	
\subsection{Malwarebytes Anti-Ransomware}
Einen anderen Weg zur Erkennung von Ransomware geht die Software "`Anti-Ransomware"' der Firma Malwarebytes. \cite{malwarebytes} \\
Sie setzt stark auf Heuristiken und versucht anhand der typischen Verhaltensmustern von Ransomware Prozess dementsprechend zu kategorisieren. 

Hierbei ist vor allem das häufige Öffnen und komplette Lesen von verschiedensten Dateien bei gleichzeitigen Schreibzugriffen in neue Dateien ein Verhalten, dass beispielsweise stark auf Ransomware hindeuten würde.

Wird ein entsprechender Prozess erkannt, wird dieser beendet, in Quarantäne verschoben und der Nutzer informiert.\\
Zu beachten ist hierbei der noch experimentelle Status der Software.



\section{Zusammenfassung}
\subsection{Aktueller Stand}

\begin{table}[]
	\centering
	\label{ransom-table}
	\caption{Aktueller Stand Ransomware}
	\begin{tabular}{|l|l|}
		\hline
\textbf{Ransomware} &                                                                                                                              \\ \hline
\textbf{}           &                                                                                                                              \\ \hline
Locky               & \cellcolor[HTML]{FFCC67}Necurs-Botnetz verschwunden.                                                                         \\ \hline
KeyRanger           & \cellcolor[HTML]{9AFF99}Neuinstallationen von Apple verhindert.                                                              \\ \hline
Petya               & \cellcolor[HTML]{FFCC67}Daten-Wiederherstellung möglich.                                                                     \\ \hline
TeslaCrypt          & \cellcolor[HTML]{9AFF99}\begin{tabular}[c]{@{}l@{}}Entwicklung eingestellt.\\ Daten-Wiederherstellung möglich.\end{tabular}  \\ \hline
CryptoLocker        & \cellcolor[HTML]{9AFF99}\begin{tabular}[c]{@{}l@{}}ZeuS-Botnetz zerschlagen.\\ Daten-Wiederherstellung möglich.\end{tabular} \\ \hline

	\end{tabular}
\end{table}

Die Sicherheitsbestrebungen der IT-Security-Dienstleister haben soweit Erfolg gezeigt und aktuelle Varianten von Ransomware konnte so Einhalt geboten werden. Aber nicht nur von Seiten der Computerindustrie wird versucht Ransomware zu stoppen, auch die Strafverfolgungsbehörden machen den Entwicklern zunehmend das Leben schwer: Durch die Festnahme von 50 russischen Hackern, so wird vermuted, wurde eines der größten Botnetze der Welt zum Stillstand gebracht\cite{angler}.

\subsection{Zukunft}

Die aktuelle Flut an Ransomware ist soweit es geht eingedämmt, wobei es hier nur eine Frage der Zeit ist, bis sich neue Varianten entwickeln und verbreiten. Noch keine Malware hatte den gleichen monetären Erfolg wie Ransomware zurzeit. Durch das Internet of Things (IoT) wird es immer mehr Geräte geben, die anfällig für Ransomware sind, vor allem in Anbetracht der Tatsache, dass die Hersteller von IoT-Geräten oftmals aus Geschäftsbereichen kommen, die mit IT und IT-Sicherheit noch keinen Kontakt hatten. Diese Firmen haben noch keine Erfahrungen mit Sicherheitslücken gemacht und werden höchstwahrscheinlich, sobald ihre Produkte auf dem Markt sind, auch keine Updates heraus bringen.\\
Wie in \glqq The Evolution Of Ransomware\grqq{} bereits angemerkt ist, werden sich die Angreifer weiter nicht nur mit Computer und Handys als Ziele beschäftigen, sondern auch andere verheißungsvolle Bereiche suchen. Mitte Juni 2016 wurde bekannt, dass die erste Ransomware für Smart-TVs entdeckt wurde\cite{smarttv}.


%\subsubsection*{Acknowledgments}


%%%%%%%%%%%%%%%%%%%%%%%%%%%%%%%%%%%%%%%%%%%%%%%%%%%%%%%%%%%%%%%%%%%%%%%%%%%%%%%
\newpage
\bibliographystyle{splncs03}
\bibliography{paper}
%%%%%%%%%%%%%%%%%%%%%%%%%%%%%%%%%%%%%%%%%%%%%%%%%%%%%%%%%%%%%%%%%%%%%%%%%%%%%%%

\end{document}
