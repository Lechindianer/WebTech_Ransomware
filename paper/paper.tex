
\documentclass[runningheads,a4paper]{llncs}

\usepackage[ngerman]{babel}

\usepackage{graphicx}

%extended enumerate, such as \begin{compactenum}
\usepackage{paralist}

%put figures inside a text
%\usepackage{picins}
%use
%\piccaptioninside
%\piccaption{...}
%\parpic[r]{\includegraphics ...}
%Text...

%Sorts the citations in the brackets
%\usepackage{cite}

%for easy quotations: \enquote{text}
\usepackage{csquotes}

\usepackage[T1]{fontenc}
\usepackage[utf8]{inputenc}

%enable margin kerning
\usepackage{microtype}

%better font, similar to the default springer font
\usepackage[%
rm={oldstyle=false,proportional=true},%
sf={oldstyle=false,proportional=true},%
tt={oldstyle=false,proportional=true,variable=true},%
qt=false%
]{cfr-lm}
%
%if more space is needed, exchange cfr-lm by mathptmx
%\usepackage{mathptmx}


\usepackage[
	%pdfauthor={},
	%pdfsubject={},
	%pdftitle={},
	%pdfkeywords={},
	bookmarks=false,
	breaklinks=true,
	colorlinks=true,
	linkcolor=black,
	citecolor=black,
	urlcolor=black,
	%pdfstartpage=19,
	pdfpagelayout=SinglePage
]{hyperref}

%enables correct jumping to figures when referencing
\usepackage[all]{hypcap}

\usepackage[capitalise,nameinlink]{cleveref}
%Nice formats for \cref
\crefname{section}{Abschnitt}{Abschnitte}
\Crefname{section}{Abschnitt}{Abschnitte}
\crefname{figure}{Abb.}{Abb.}
\Crefname{figure}{Abbildung}{Abbildungen}

\usepackage{xspace}
%\newcommand{\eg}{e.\,g.\xspace}
%\newcommand{\ie}{i.\,e.\xspace}
\newcommand{\eg}{e.\,g.,\ }
\newcommand{\ie}{i.\,e.,\ }

%introduce \powerset - hint by http://matheplanet.com/matheplanet/nuke/html/viewtopic.php?topic=136492&post_id=997377
\DeclareFontFamily{U}{MnSymbolC}{}
\DeclareSymbolFont{MnSyC}{U}{MnSymbolC}{m}{n}
\DeclareFontShape{U}{MnSymbolC}{m}{n}{
    <-6>  MnSymbolC5
   <6-7>  MnSymbolC6
   <7-8>  MnSymbolC7
   <8-9>  MnSymbolC8
   <9-10> MnSymbolC9
  <10-12> MnSymbolC10
  <12->   MnSymbolC12%
}{}
\DeclareMathSymbol{\powerset}{\mathord}{MnSyC}{180}

%improve wrapping of URLs - hint by http://tex.stackexchange.com/a/10419/9075
\makeatletter
\g@addto@macro{\UrlBreaks}{\UrlOrds}
\makeatother

% correct bad hyphenation here
\hyphenation{op-tical net-works semi-conduc-tor}

\begin{document}

%Works on MiKTeX only
%hint by http://goemonx.blogspot.de/2012/01/pdflatex-ligaturen-und-copynpaste.html
%also http://tex.stackexchange.com/questions/4397/make-ligatures-in-linux-libertine-copyable-and-searchable
%This allows a copy'n'paste of the text from the paper
\input glyphtounicode.tex
\pdfgentounicode=1

\title{Web Technologien: Ransomware}
%If Title is too long, use \titlerunning
\titlerunning{Web Technologien: Ransomware}

%Single insitute
\author{Abwandner, Sebastian und Beham, Michael und Schmid, Pascal}
%If there are too many authors, use \authorrunning
%\authorrunning{First Author et al.}
\institute{
	Hochschule Augsburg \\
	An der Hochschule 1 \\
	86161 Augsburg
}


\maketitle

\begin{abstract}
Das Thema Ransomware war in
\end{abstract}

%%%%%%%%%%%%%%%%%%%%%%%%%%%%%%%%%%%%%%%%%%%%%%%%%%%%%%%%%%%%%%%%%%%%%%%%%%%%%%%
\section{Einleitung}\label{sec:intro}
%%%%%%%%%%%%%%%%%%%%%%%%%%%%%%%%%%%%%%%%%%%%%%%%%%%%%%%%%%%%%%%%%%%%%%%%%%%%%%%

\section{Geschichte}

\section{Einteilung}
Virentypen usw.

\section{Aktuelle Ransomware}

\section{Faktoren von Ransomware}

\section{Kryptographie}

\section{Zahlung}
Test


\section{Tor}

\section{Prävention und Gegenmaßnahmen}
BSI \cite{BSI:Ransomware}


\section{Zusammenfassung}


%\subsubsection*{Acknowledgments}


%%%%%%%%%%%%%%%%%%%%%%%%%%%%%%%%%%%%%%%%%%%%%%%%%%%%%%%%%%%%%%%%%%%%%%%%%%%%%%%
\bibliographystyle{splncs03}
\bibliography{paper}

All links were last followed on December 14, 2015.
%%%%%%%%%%%%%%%%%%%%%%%%%%%%%%%%%%%%%%%%%%%%%%%%%%%%%%%%%%%%%%%%%%%%%%%%%%%%%%%

\end{document}
