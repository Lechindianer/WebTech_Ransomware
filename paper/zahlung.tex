Eine weitere Komponente warum heutige Ransomware so erfolgreich ist, stellt die Möglichkeit einer anonymen Bezahlung dar, vor allem in Hinsicht auf die Anonymität des Zahlungsempfängers. \\
Waren beim BKA-Trojaner noch kartengebundene Zahlungssysteme wie Paysafecard und Ukash im Einsatz, verlagerte sich die Zahlung bei heutiger Crypto-Ransomware vor allem auf Bitcoin. Dies dürfte auch mit der heutigen Popularität von Bitcoins und dem Erreichen breiter Schichten der Bevölkerung. Trotzdem bietet beispielsweise Petya eine genau Erklärung zum Erwerb und zur Weitergabe der Bitcoins, um auch unkundige Nutzer die Zahlungen zu ermöglichen. \cite{petya:infect}

\subsection{Paysafecard}
Paysafecard ist ein anonymes Zahlungsmittel, bei dem der Kunde an einer Verkaufsstelle (z.B. Automat, Tankstelle, Einzelhandel) eine Guthabenkarte mit einem Wert zwischen 10 € und 100 € erwirbt. Das Guthaben kann mittels der Kenntnis einer aufgedruckten PIN weitergeben und ausbezahlt werden. \\
Durch die fehlende Kontrolle beim Erwerb der Karten und der Möglichkeit die Guthaben automatisiert europaweit auszubezahlen, war für die Angreifer eine hinreichende Anonymität anscheinend sichergestellt. 

Seitens der Politik wurde immer wieder diskutiert, ob derartige Zahlungssysteme nicht reguliert oder verboten werden sollten, allerdings konnte hierüber bis heute keine abschließende Einigung erzielt werden. \cite{paysafecard}

Bei moderner Ransomware spielen guthaben-basierte Zahlungsmittel keine Rolle mehr.

\subsection{Bitcoin}
Bitcoin wird von moderner Crypto-Ransomware als bevorzugtes Zahlungsmittel verwendet. Bitcoin wurde von Anfang an als dezentrales, anonymes Zahlungsmittel konstruiert. Hinsichtlicher der Nutzung durch Ransomware bietet sich hier der Vorteil, dass sowohl die Adressen (vergleichbar mit einem Bankkonto) ohne zentrale Stelle automatisiert erzeugt werden können, als auch die Möglichkeit mit Hilfe von Schnittstellen diese Adressen ebenfalls automatisiert auf Zahlungseingänge überwachen zu können. 

Typischerweise wird für jeden "`Nutzer"' der Ransomware eine eigene Bitcoin-Adresse erzeugt, um die Zahlungen eindeutig zuordnen zu können und um eine Nachverfolgung der Zahlungen, die bei Bitcoin prinzipbedingt öffentlich sichtbar sind, zu erschweren. 

Für weiterführende Informationen bezüglich Bitcoin empfiehlt sich das Paper von Satoshi Nakamoto. \cite{bitcoin}
