\subsection{Aktuelle Patchlevel}
	 Grundsätzlich sollte, um die allgemeine Sicherheit eines Systems zu verbessern, immer jegliche Software, sowohl des Betriebssystem wie auch die verwendete Anwendungssoftware, auf den aktuellen Stand gehalten werden. Ein besonders großes Risiko besteht bei Software, die zum öffnen von Inhalten aus den Internet verwendet werden, wie z.B. Web-Browser, Browser-Plugins, E-Mail-Programme, PDF-Betrachter und Büroanwendungen. \cite{bsi:ransome}[Seite 11, 4.1.1 Patches]\\
	 
	 Dies gilt in besonderem Maße für Systeme auf denen Windows betrieben wird, da diese für die Erpresser die größte Verbreitung ermöglichen, was einfach der großen Verbreitung des Betriebssystems zuzuschreiben ist. Dies soll keinesfalls bedeuten, dass alternative Systeme weniger gefährdet sind, allerdings das Risiko bei Windows basierten Systemen erhöht sein könnte.
\subsection{Angriffsfläche minimieren}
	Zudem kann die Angriffsfläche reduziert werden, indem nicht benötigt Software deinstalliert wird, wodurch auf die Anzahl der Software, die auf dem neuesten Stand gehalten werden muss sich verringert.\\
	
	Des Weiten ist kann das Systemweite deaktivieren der Script-Ausführung, wenn dies möglich ist das Risiko deutlich minderen, da Ransomeware teilweise über E-Mail-Anhänge in Form von Javascript und VisualBasic-Skripten verteilt wird. Somit kann der Anhang nicht auf dem System ausgeführt werden und eine Infektion wird verhindert. \cite{bsi:ransome}[Seite 11, 4.1.2 Angriffsfläche minimieren]\\
	
	Zudem ist es sinnvoll, wenn nicht benötigte Browser-Plugins deinstalliert werden (z.B. Flash, Java, Silverlight)\cite{bsi:ransome}[Seite 11, 4.1.2 Angriffsfläche minimieren]\\, da diese bekanntermaßen häufig über Sicherheitslücken verfügen, was allein schon an den häufigen Sicherheits-Pachtes festgemacht werden kann.  
\subsection{Makros deaktivieren}
	Makros in z.B Excel können grundsätzlich ein Sicherheitsrisiko darstellen. Ransomeware kann sich z.B. unscheinbar wirkenden Excel, oder ähnlichen Dateien, als Makro verstecken. Bei aktuellen Office-Produkten ist das automatische ausführen von Makros zwar deaktiviert, in älteren allerdings nicht. Bei solch älterer Software ist es zudem häufig, dass diese nicht mehr mit Sicherheits-Updates versorgt werden, somit ist von der Verwendung sowieso abzuraten. 
\subsection{Text-Mail erzwingen}
	Dem BSI zufolge ist ist eine eine gute Idee, die HTML-Darstellung von E-Mails zu deaktivieren. Dies kann hilfreich sein, da für die Darstellung solcher E-Mails die selben Mechanismen benötigt werden, wie sie in Webbrowsern nötig sind. Dies bringt weiter Komplexität in die E-Mail Software und somit potentiell weitere Sicherheitslücken. Die können natürlich nicht ausgenutzt werden, wenn man entsprechende Komponente nicht nutzt. Zudem sollte man das automatische ausführen von Anhängen beim anklicken verhindern.\cite{bsi:ransome}[Seite 11, 4.1.3 Behandlung von E-Mails]
\subsection{Serverseitige Mailfilter}
	
\subsection{Netzlaufwerke sichern}
\subsection{Netzwerke segmentieren}
\subsection{Zugänge sichern}
\subsection{Aktueller Vierenschutz}
\subsection{Nutzerschulung}
\subsection{Backups / Datensicherungen}
