Aufgrund der sich teilweise überschneidenden Funktionen im Bereich der Malware und der teils sehr unscharfen oder auch falschen Verwendung der Worte in Medien aller Art ist es schwierig eine allgemeingültige Definition dafür anzugeben. Im Rahmen dieser Arbeit wird sich jedoch an die Definitionen des Microsoft Protection Centers \cite{malware_pc} gehalten.

		\begin{itemize}
			\item \textbf{Bot:} Ein kleines, verstecktes Programm auf dem Rechner des Benutzers, oftmals vom Angreifer kontrolliert. Mehrere Bots, die über das Internet verbunden sind, nennt man ein Botnetz.
			\item \textbf{Botnetz:} Mehrere Kopien des­sel­ben Bots sind auf mehreren Rechnern installiert und werden von einem Angreifer kontrolliert. Mehrere Bots auf einer großen Anzahl infizierter Rechner nennt man Botnetz.
			\item \textbf{Malware:}	Kurzform für ``malicious software''. Überbegriff für Software, die ungewollte Aktivitäten auf einem Rechner durchführt, wie etwa Daten stehlen, Dateien verschlüsseln oder Spam verschickt.
			\item \textbf{Ransomware:} Eine Art Malware, die den Benutzer daran hindert, das Gerät zu benutzen oder die Dateien verschlüsselt. Der Benutzer wird darüber informiert, dass er Geld bezahlen, eine Befragung oder eine sonstige Aktion durchführen muss, bevor er die Kontrolle wiedererlangt.
			\item \textbf{Trojaner:} Nach dem Trojanischen Pferd aus der Sage benannt, verhält sich diese Art von Malware unauffällig, um nicht entdeckt zu werden. Im Gegensatz zu Würmern oder Viren verbreitet sich ein Trojaner nicht von selbst weiter, sondern täuscht den Benutzer, um von diesem heruntergeladen oder installiert zu werden.
			\item \textbf{Virus:} Malware, die sich selbst weiterverbreitet, indem es reguläre Programme befällt und sich auf andere Systeme und Netzwerke weiterverbreitet.
			\item \textbf{Wurm:} Ein Wurm unterscheidet sich vom Virus dadurch, dass er mithilfe eines Trägers verbreitet wird. Dies kann etwa durch E-Mails, Instant Messaging, File-Sharing, Netzwerklaufwerke, Wechsellaufwerke und Ähnliches geschehen.
		\end{itemize}
Diese Einteilung gibt einen kurzen Überblick über die gebräuchlichsten Begrifflichkeiten, ist aber nicht abschließend. \\
Einer klaren Kategorisierung steht auch entgegen, dass Schädlinge meist in mehrere Kategorien fallen und sich diese auch teilweise überschneiden.