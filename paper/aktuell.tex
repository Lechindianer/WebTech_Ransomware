Alle hier beschriebenen Ransomware sind innerhalb des letzten Jahres erschienen und beschreiben somit den zum Schreiben des vorliegenden Papers aktuellen Stand der Technik.

\subsection{TeslaCrypt}
TeslaCrypt war der erste der modernen Ransomware, der von den Medien aufgegriffen wurde und somit der Öffentlichkeit bekannt gemacht wurde. Ende 2015\cite{tesla:entdeckt} wurde dieser entdeckt, jedoch wurde bereits kurze Zeit später ein Tool veröffentlicht\cite{tesla:geknackt}, mit dem betroffene Anwender ihre Daten retten konnten.
Version 3 der Ransomware wurde von den Angreifern grundlegend verändert. So wurde der Algorithmus zum Schlüsselaustausch erneuert, was es vorerst unmöglich gemacht hat, den privaten Schlüssel der Ransomware aus den verschlüsselten Dateien herzustellen\cite{tesla:version3}\cite{tesla:version3_2}.
Die letzte Version 4 kann zudem Dateien größer als 4 GB verschlüsseln\cite{tesla:version4}.....
		
\subsection{Petya}
\subsection{Locky}
\subsection{KeRanger}