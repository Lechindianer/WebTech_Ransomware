Alle hier beschriebenen Ransomware sind innerhalb des letzten Jahres erschienen und beschreiben somit den zum Schreiben des vorliegenden Papers aktuellen Stand der Technik.

\subsection{TeslaCrypt}
TeslaCrypt war der erste der modernen Ransomware, der von den Medien aufgegriffen wurde und somit der Öffentlichkeit bekannt gemacht wurde. Ende 2015\cite{tesla:entdeckt} wurde dieser entdeckt, jedoch wurde bereits kurze Zeit später ein Tool veröffentlicht\cite{tesla:geknackt}, mit dem betroffene Anwender ihre Daten retten konnten.\\
Version 3 der Ransomware wurde von den Angreifern grundlegend verändert. So wurde der Algorithmus zum Schlüsselaustausch erneuert, was es vorerst unmöglich gemacht hat, den privaten Schlüssel der Ransomware aus den verschlüsselten Dateien herzustellen\cite{tesla:version3}\cite{tesla:version3_2}.\\
Version 4 kann zudem Dateien größer als 4 GB verschlüsseln\cite{tesla:version4} --- zuvor wurden Dateien größer als dieses Limit unwiderruflich zerstört. In der vorerst letzten veröffentlichen Version 4.1A\cite{tesla:version41} wurden weitere Dateitypen verschlüsselt. So wurden unter anderem Bitcoin-Wallets und Spielstände verschlüsselt. Um einer Analyse zu entgehen, wurde die Möglichkeit die Malware während des Betriebs zu detektieren dadurch verhindert, dass man die Malware im Task-Manager und dem Process Explorer von SysInternals nicht mehr sieht.\\
Am 18. Mai 2016 wurde bekannt\cite{tesla:end}, dass die Entwickler von TeslaCrypt die Weiterentwicklung eingestellt haben und als Entschuldigung den Schlüssel zur Verschlüsselung frei auf ihrer Webseite veröffentlicht haben. Die Wochen zuvor wurde von den Experten der IT-Sicherheitsfirma ESET bereits festgestellt, dass die Entwickler der Ransomware ihre Bestrebungen immer weiter verringern.

		
\subsection{Petya}
\subsection{Locky}
\subsection{KeRanger}