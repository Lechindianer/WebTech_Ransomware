\subsection{Aktueller Stand}

\begin{table}[]
	\centering
	\label{ransom-table}
	\caption{Aktueller Stand Ransomware}
	\begin{tabular}{|l|l|}
		\hline
\textbf{Ransomware} &                                                                                                                              \\ \hline
\textbf{}           &                                                                                                                              \\ \hline
Locky               & \cellcolor[HTML]{FFCC67}Necurs-Botnetz verschwunden.                                                                         \\ \hline
KeyRanger           & \cellcolor[HTML]{9AFF99}Neuinstallationen von Apple verhindert.                                                              \\ \hline
Petya               & \cellcolor[HTML]{FFCC67}Daten-Wiederherstellung möglich.                                                                     \\ \hline
TeslaCrypt          & \cellcolor[HTML]{9AFF99}\begin{tabular}[c]{@{}l@{}}Entwicklung eingestellt.\\ Daten-Wiederherstellung möglich.\end{tabular}  \\ \hline
CryptoLocker        & \cellcolor[HTML]{9AFF99}\begin{tabular}[c]{@{}l@{}}ZeuS-Botnetz zerschlagen.\\ Daten-Wiederherstellung möglich.\end{tabular} \\ \hline

	\end{tabular}
\end{table}

Die Sicherheitsbestrebungen der IT-Security-Dienstleister haben soweit Erfolg gezeigt und aktuelle Varianten von Ransomware konnte so Einhalt geboten werden. Aber nicht nur von Seiten der Computerindustrie wird versucht Ransomware zu stoppen, auch die Strafverfolgungsbehörden machen den Entwicklern zunehmend das Leben schwer: Durch die Festnahme von 50 russischen Hackern, so wird vermuted, wurde eines der größten Botnetze der Welt zum Stillstand gebracht\cite{angler}.

\subsection{Zukunft}

Die aktuelle Flut an Ransomware ist soweit es geht eingedämmt, wobei es hier nur eine Frage der Zeit ist, bis sich neue Varianten entwickeln und verbreiten. Noch keine Malware hatte den gleichen monetären Erfolg wie Ransomware zurzeit. Durch das Internet of Things (IoT) wird es immer mehr Geräte geben, die anfällig für Ransomware sind, vor allem in Anbetracht der Tatsache, dass die Hersteller von IoT-Geräten oftmals aus Geschäftsbereichen kommen, die mit IT und IT-Sicherheit noch keinen Kontakt hatten. Diese Firmen haben noch keine Erfahrungen mit Sicherheitslücken gemacht und werden höchstwahrscheinlich, sobald ihre Produkte auf dem Markt sind, auch keine Updates heraus bringen.\\
Wie in \glqq The Evolution Of Ransomware\grqq{} bereits angemerkt ist, werden sich die Angreifer weiter nicht nur mit Computer und Handys als Ziele beschäftigen, sondern auch andere verheißungsvolle Bereiche suchen. Mitte Juni 2016 wurde bekannt, dass die erste Ransomware für Smart-TVs entdeckt wurde\cite{smarttv}.