\subsection{Aktueller Stand}

\begin{table}[]
	\centering
	\label{ransom-table}
	\caption{Aktueller Stand Ransomware}
	\begin{tabular}{|l|l|}
		\hline
\textbf{Ransomware} &                                                                                                                              \\ \hline
\textbf{}           &                                                                                                                              \\ \hline
Locky               & \cellcolor[HTML]{FFCC67}Necurs-Botnetz verschwunden.                                                                         \\ \hline
KeyRanger           & \cellcolor[HTML]{9AFF99}Neuinstallationen von Apple verhindert.                                                              \\ \hline
Petya               & \cellcolor[HTML]{FFCC67}Daten-Wiederherstellung möglich.                                                                     \\ \hline
TeslaCrypt          & \cellcolor[HTML]{9AFF99}\begin{tabular}[c]{@{}l@{}}Entwicklung eingestellt.\\ Daten-Wiederherstellung möglich.\end{tabular}  \\ \hline
CryptoLocker        & \cellcolor[HTML]{9AFF99}\begin{tabular}[c]{@{}l@{}}ZeuS-Botnetz zerschlagen.\\ Daten-Wiederherstellung möglich.\end{tabular} \\ \hline

	\end{tabular}
\end{table}

Die Sicherheitsbestrebungen der IT-Security-Dienstleister haben soweit Erfolg gezeigt und aktuelle Varianten von Ransomware konnte so Einhalt geboten werden. Aber nicht nur von Seiten der Computerindustrie wird versucht Ransomware zu stoppen, auch die Strafverfolgungsbehörden machen den Entwicklern zunehmend das Leben schwer: Durch die Festnahme von 50 russischen Hackern, so wird vermuted, wurde eines der größten Botnetze der Welt zum Stillstand gebracht \cite{angler}.

\subsection{Zukunft}

Die aktuelle Flut an Ransomware ist soweit es geht eingedämmt, wobei es hier nur eine Frage der Zeit ist, bis sich neue Varianten entwickeln und verbreiten. Noch keine Malware hatte den gleichen finanziellen Erfolg wie Ransomware zurzeit. Durch das Internet of Things (IoT) wird es zukünftig immer mehr Geräte geben, die anfällig für Ransomware sind, vor allem in Anbetracht der Tatsache, dass die Hersteller von IoT-Geräten oftmals aus Geschäftsbereichen kommen, die mit IT und IT-Sicherheit noch keinen Kontakt hatten. Diese Firmen haben noch keine Erfahrungen mit Sicherheitslücken gemacht und werden höchstwahrscheinlich, sobald ihre Produkte auf dem Markt sind, auch keine Updates heraus bringen. Diese Praxis ist vor allem bei Embedded-Systemen zu beobachten, beispielsweise bei Routern, Fernsehern und ähnlichen Produkten.\\
Die Hersteller verweisen zwar gerne auf die Update-Pflicht der Nutzer, oftmals ist die Update-Prozedur allerdings so kompliziert, dass es fragwürdig ist, ob Nutzer ohne tiefer gehende Kenntnisse dazu in der Lage sind.

Ebenfalls kritisch zu sehen ist die großflächige Verbreitung von Android als Monokultur auf verschiedensten Typen von Geräten, angefangen von Smartphones über Fernsehgeräte und Festnetztelefone, die ähnlich Microsoft Windows aufgrund der rein zahlenmäßig großen Verbreitung die Entwicklung von Schadsoftware als lohnendes Ziel erscheinen lässt. So wurde Mitte Juni 2016 eine erste Ransomware für einen Smart-TV entdeckt \cite{smarttv}.

Auch von der Qualität der Verschlüsselung zeigt sich eine deutliche Entwicklung hin zur Professionalisierung. War erste Crypto-Ransomware noch mit selbst implementierten Algorithmen zur Schlüsselerzeugung und Verschlüsselung, die aufgrund der komplexen und komplizierten Implementierung oftmals essentielle Fehler enthielt, ausgestattet, so greifen die Entwickler nun eher zu fertigen Krypto-Lösungen mit getesteten Implementierungen. Als Beispiel für die Auswirkungen einer mangelhaften Verschlüsselung wäre Petya zu nennen, hier machte u.a. eine zu kurze Wortbreite eine Entschlüsselung möglich. \cite{petya:decrypt}